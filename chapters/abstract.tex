%!TEX root = ../main.tex
%%%%%%%%%%%%%%%%%%%%%%%%%%%%%%%%%%%%%%%%%%%%%%%%%%%%%%%%%%%%%%%%%%%%
%% abstrac-en.tex
%% NOVA thesis document file
%%
%% Abstract/Summary
%%%%%%%%%%%%%%%%%%%%%%%%%%%%%%%%%%%%%%%%%%%%%%%%%%%%%%%%%%%%%%%%%%%%

\typeout{NT FILE abstrac.tex}%

Regardless of the language in which the dissertation is written, usually there are at least two abstracts: 
one abstract in the same language as the main text, and another abstract in some other language.

The abstracts' order varies with the school. If your school has specific regulations concerning the abstracts' 
order, the \gls{novathesis} (\LaTeX) template will respect them.  Otherwise, the default rule in the 
\gls{novathesis} template is to have in first place the abstract in \emph{the same language as main text}, 
and then the abstract in \emph{the other language}. For example, if the dissertation is written in Portuguese, 
the abstracts' order will be first Portuguese and then English, followed by the main text in Portuguese. If 
the dissertation is written in English, the abstracts' order will be first English and then Portuguese, 
followed by the main text in English.
%
However, this order can be customized by adding one of the following to the file \verb+5_packages.tex+.

\keywords{
  Cathegory Theory \and
  Lambda Calculus \and
  Haskell \and
  Mathematical Model \and
  Axiomatic Method
}
